\documentclass{extension}

%\usepackage{titling}
%\setlength{\droptitle}{10mm}

\usepackage[superscript]{cite}
\usepackage{hyperref}
\makeatletter
\renewcommand\@citess[1]{\textsuperscript{[#1]}}
\renewcommand\refname{Materials}

\newcommand{\mm}{\,$\mathrm{mm}$}
\newcommand{\m}{\,$\mathrm{m}$}
\newcommand{\uunit}{\,$\mathrm{W\,m^{-2}\,K}$}

\begin{document}

\title{Building Regulations Application \\[2mm]
Supporting Documentation}
\author{J. Mullaney}
\date{\today}

\maketitle

\vspace{5mm}
\section{General summary}
Alteration to existing extension and garage of single semi-detached dwellinghouse. Construction of porch.

\section{Current property}
The property is a 1940s-build three-bedroomed semi-detached dwellinghouse. At the time of construction, it consisted of the main house with a standalone garage. During the 1970’s a single-storey extension was built in the space between the main house and the garage; the extension is approximately 0.5 m taller than the garage. The garage and extension are set-back from the main house by approximately 1.6\,m.

\section{Summary of proposed alterations}
Proposed is a series of modest alterations to the front and interior of the property:
\begin{enumerate}
  \item A small porch with a footprint of 1.3\m$\times$2.3\m\ (area $=3\,{\rm m^2}$) and a height of 3.13\m\ will be added to the front of the property; see \S\ref{porch});
  \item The front elevation of the existing extension and garage will be brought in-line with the front elevation of the main property. A new doorway in a load bearing wall will give access to the extension; see \S\ref{extension};\\
  \item A utility room will be built at the back of the existing garage. A new doorway in a load-bearing wall will give access to the utility room; see \S\ref{utility}.
\end{enumerate}

Detailed drawings with accompanying structural and U-value calculations are included with this application. What follows is a detailed description of the proposed build, including references -- in square brackets -- to details of and links to the materials.

\subsection{Porch}
\label{porch}
With a footprint of less than $=30\,{\rm m^2}$, and requiring no structural changes to the existing structure, building regulations approval is not required for the proposed small porch; it is only included here for completeness.
\subsubsection{Structural}
\begin{itemize}
  \item {\bf Impact on existing structure:} The porch will lead to the existing front door, so its construction will not involve any structural work on the existing property.
  \item {\bf Walls:} The external walls will consist of a single skin of clay bricks and 7.3\,N standard Dense Concrete Blocks (DCBs), tied to the existing building every 225\mm\ vertically. A Damp Proof Course (DPC) will run at least 150\,mm above the ground level.
  \item{\bf Footings:} The walls will sit on 900\mm\ deep, 225\mm\ thick, 500\mm\ wide concrete strip footings.
  \item {\bf Openings:} A single-leaf steel lintel with a Safe Working Load (SWL) of 5\,kN\cite{psteel} will support 0.4\,kN load above the new front door. A concrete lintel\cite{pconc} will support the max. 0.4\,kN load above the window.
  \item {\bf Roof:} A cold flat roof will be supported by C16 joists. One end of the joists will rest on the new wall, to which the joists will be tied, with the other end supported by joist hangers attached to a C16 ledger running the length of the porch. The ledger will be bolted to the existing stucture using M12 bolts resin-fixed to the existing structure at max. 400\mm\ horizontal spacings.
\end{itemize}
\subsubsection{Heat and Sound Insulation}
While the porch will not be heated, it will be insulated. Details follow:
\begin{itemize}
  \item {\bf Floor:} A 50\mm\ layer of Extanded Polystyrene (EPS)\cite{pfins} insulation will insulate the concrete floor from the ground.
  \item {\bf Walls:} Brick walls will be lined internally with studwork which will be in-filled with 100\mm\ EPS insulation.
  \item {\bf Windows:} The single window will be double-glazed with uPVC frames.
  \item {\bf Roof:} The inter-joist space will be filled with 100\mm\ of mineral wool insulation.\cite{prins}
\end{itemize}
\subsubsection{Water runoff}
\begin{itemize}
  \item {\bf Roof:} 20:1 firring strips will be used to create a slope on the flat roof. A 1.2\mm\ thick EPDM membrane\cite{prmem} fitted according to manufacturer guidelines will be used to waterproof the roof.
  \item {\bf Drainage:} Rainwater will drain into a gutter that will ultimately flow into an existing drain next to the garage.
\end{itemize}
\subsubsection{Utilities}
\begin{itemize}
  \item {\bf Electrical:} No electrical sockets are planned for the porch. A single internal electrical internal light will be installed in the ceiling, and downward-directed external lights will be installed externall. All electrical light fittings will be installed or approved by a qualified electrician.
  \item {\bf Heating:} The porch will not be heated.
  \item {\bf Water:} N/A
\end{itemize}

\subsection{Extension}
\label{extension}
The alterations to the existing extension will involve the demolition of the front-facing wall and the construction of three walls: one parallel to the front elevation of the house (labelled as Cutaway A on plans), and two parallel to the side elevation. One of the latter will extend the existing wall between the extension and the garage (Cutaway B), the other will extend the external garage wall (Cutaway C).
\subsubsection{Structural}
\begin{itemize}
  \item {\bf Impact on existing structure:} A new 832\mm -wide doorway will be cut in an existing load-bearing wall to give access to the extension from the existing hallway. There are no point loads or openings above the planned doorway, and the first floor joists of the existing structure are above the 45$^\circ$ load triangle. A 1200\mm\ StressLine SL90 Steel Lintel rated to 17\,kN will support the 0.64\,kN load above the opening. The lintel will rest on 180\mm -wide bearings with padstones used if required.
  \item {\bf Walls:} The walls of the extension will consist of a double skin of clay brick and 7.3\,N DCBs tied to the existing building every 225\mm\ vertically. The inner and outer skins will be tied together every 600\mm\ horizontally and 450\mm\ vertically, reducing to 225\mm\ around openings. The garage wall will consist of a single skin of clay brick and 7.3\,N DCBs tied to the existing building every 225\mm\ vertically. The single skin wall will have a 327$\times$215\mm\ pier at the end, which is 1300\mm\ from a pier in the existing garage wall. All walls will have separate DPCs for the inner and outer skins at least 150\mm\ above the ground level.
  \item {\bf Footings:} The double-skin walls will sit on 900\mm\ deep, 225\mm\ thick, 600\mm\ wide concrete strip footings. The single-skin wall will sit on 900\mm\ deep, 225\mm\ thick, 400\mm\ wide concrete strip footings.
  \item {\bf Openings:} The front wall of the exension will include a large opening to accommodate a near floor-to-ceiling window. The opening will extend upward to the roof joists, so no load-bearing lintel is required.
  \item {\bf Roof:} A warm flat roof will be supported by C16 joists. One end of the joists will rest on and be tied to the inner leaf of the exended wall between the extension and the garage. The other end of the joists will be supported by joist hangers attached to a C16 ledger running the entire length of the extension. The ledger will be bolted to the existing stucture using M12 bolts resin-fixed to the existing structure at max. 400\mm\ horizontal spacings.

\end{itemize}

\subsubsection{Heat and Sound Insulation}
\begin{itemize}
  \item {\bf Floor:} A 100\mm\ layer of Extanded Polystyrene (EPS)\cite{efins} insulation will insulate the concrete floor from the ground. The overall U-value of the floor is 0.15\uunit .
  \item {\bf Walls:} The 140\mm\ front-facing wall cavity will be partially filled with a 115\mm\ layer of Cavity Board Insulation.\cite{ew1ins} The overall U-value of the front-facing wall is 0.15\uunit . The 85\mm\ cavity in the wall separating the extension from the garage will be fully-filled with mineral wool.\cite{ew2ins} A wider cavity cannot be created as the wall is extending an existing wall forward. The overall U-value of the extension/garage wall is 0.33\uunit. The single skin garage wall will not be insulated.
  \item {\bf Windows:} The window will be double glazed and have a uPVC frame. Unusually, the window will sit on the inner skin wall to accommodate the external louvres shown on the plans. This potentially introduces a thermal bridge which will be overcome by surrounding the window with CompacFoam.\cite{cfoam}
  \item {\bf Roof:} The warm flat roof will contain a 150\mm -thick layer of thermaset insulation.\cite{erins} The overall U-value of the roof is 0.15\uunit .
\end{itemize}
\subsubsection{Water runoff}
\begin{itemize}
  \item {\bf Roof:} 20:1 firring strips will be used to create a slope on the flat roof. A 1.2\mm\ thick EPDM membrane\cite{prmem} fitted according to manufacturer guidelines will be used to waterproof the roof.
  \item {\bf Drainage:} Rainwater will drain into a gutter that will ultimately flow into an existing drain next to the garage.
\end{itemize}
\subsubsection{Utilities}
\begin{itemize}
  \item {\bf Electrical:} The existing ring-main will be extended to accommodate two double plug sockets. The existing light switch will be moved to the new entrance spotlights will be installed in the ceiling. All electrical light fittings will be installed or approved by a qualified electrician.
  \item {\bf Heating:} The extension will be heated. The current wall-mounted radiator will be replaced by a larger one.
  \item {\bf Water:} There will be no mains water supply in the extension.
\end{itemize}

\subsection{Utility Room \& WC}
\label{utility}


\begin{thebibliography}{9}
\bibitem{psteel} \href{https://www.travisperkins.co.uk/steel-lintels/catnic-external-solid-wall-single-leaf-angle-lintel-1200mm-ang1200/p/270812}{Catnic External Solid Wall Single Leaf Angle Lintel ANG BSD1002700 $88\times91\times1200$\,mm}.

\bibitem{pconc} \href{https://www.travisperkins.co.uk/concrete-lintels/supreme-prestressed-textured-concrete-lintel-65mm-x-100mm-x-900mm-p100/p/700507}{Supreme Prestressed Textured Concrete Lintel $65\times100\times900$\,mm P100}

\bibitem{pfins} \href{https://www.travisperkins.co.uk/insulation-board/50mm-x-2400mm-x-1200mm-celotex-pir-insulation-board-ga4000/p/778040}{50\,mm Celotex PIR Insulation Board GA4000. Thermal conductivity: 0.022 W/mK}

\bibitem{pwins} \href{https://www.travisperkins.co.uk/insulation-board/100mm-x-2400mm-x-1200mm-celotex-pir-insulation-board-ga4000/p/778048}{100\,mm Celotex PIR Insulation Board GA4000. Thermal conductivity: 0.022 W/mK}

\bibitem{prins} \href{https://www.travisperkins.co.uk/loft-insulation/knauf-insulation-omnifit-insulation-roll-6860mm-x-1200mm-x-100mm/p/734020}{100\mm\ Knauf Insulation OmniFit Insulation Roll. Thermal conductivity: 0.04 W/mK}

\bibitem{prmem} \href{https://www.rubber4roofs.co.uk/classicbond-one-piece-epdm-rubber-roof-covering-1-20mm}{ClassicBond 1.2mm EPDM Rubber Roof Membrane}

\bibitem{efins} \href{https://www.kingspan.com/gb/en/products/insulation-boards/floor-insulation-boards/thermafloor-tf70/}{Kingspan 100mm Thermafloor TF70. Thermal conductivity: 0.022 W/mK}

\bibitem{ew1ins} \href{https://www.jewson.co.uk/p/kingspan-kooltherm-k106-cavity-board-1200-x-450-x-115mm-pack-of-4-IK106115} {Kingspan 115mm Kingspan Kooltherm K106 Cavity Board. Thermal conductivity: 0.019 W/mK}

\bibitem{ew2ins} \href{https://www.travisperkins.co.uk/cavity-and-internal-wall-insulation/knauf-dritherm-cavity-slab-32-85mm-455-x-1200mm-2-73m2-per-pack/p/848626?gclid=Cj0KCQjw7aqkBhDPARIsAKGa0oIWRlMly_UsJOFDSrQJAOasgPovH9dFJJFFpmnhtPwcoliCuB8_0a4aAik4EALw_wcB&gclsrc=aw.ds} {Knauf Dritherm Cavity Slab 32 85mm. Thermal conductivity: 0.032 W/mK}

\bibitem{cfoam} \href{https://www.greenbuildingstore.co.uk/products/compacfoam-200/} {CompacFoam 200}

\bibitem{erins} \href{https://www.kingspan.com/gb/en/products/insulation-boards/roof-insulation-boards/thermaroof-tr27/?s=t} {150\mm\ Kingspan Thermaroof TR27 insulation. Thermal conductivity: 0.024 W/mk}

\end{thebibliography}
\end{document}
