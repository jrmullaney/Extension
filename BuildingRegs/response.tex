\documentclass{extension}

%\usepackage{titling}
%\setlength{\droptitle}{10mm}

\usepackage[superscript]{cite}
\usepackage{hyperref}
\makeatletter
\renewcommand\@citess[1]{\textsuperscript{[#1]}}
\renewcommand\refname{Materials}

\newcommand{\mm}{\,$\mathrm{mm}$}
\newcommand{\m}{\,$\mathrm{m}$}
\newcommand{\uunit}{\,$\mathrm{W\,m^{-2}\,K}$}

\begin{document}

\title{Building Regulations Application \\[2mm]
Respone to schedule of additional \\ information/details for: 23/01851/FP}
\author{J. Mullaney}
\date{\today}

\maketitle

\vspace{5mm}

\section{Structure}
\begin{enumerate}
\item {\it Foundations to be to the approval of the District Building Surveyor}\\
I confirm that foundatations for the Porch and Extension will be inspected and approved by the District Building Surveyor prior to the commencement of any further construction. No new foundations are needed for the Utility Room and WC.
\item {\it New walls should be suitably bonded to existing walls. i.e., using a proprietary wall starter kit.}\\
I confirm that new walls will be bonded to the existing walls using a wall starter kit such as SabreFix Stainless Steel Wall Starter Kit, or similar.\cite{starterkit}
\item {\it Wall plates should be strapped to new masonry walls every 2m, using galvanised steel straps at least 1m length.}\\
The only reference to wall plates was on the plans for the porch. It has since been realised that they are not required, and so have been removed from the plans.
\item {\it Can you confirm the roof span? The proposed 120$\times$47 C16 joists appear slightly under spanned according to the Trada tables, at 2.4\,m maximum clear span.}\\
Thank you for pointing out this oversight. The clear span of the extension (study) is 2.36\,m, while the clear span of the rear room is 2.83\,m. So while the extension is just within the maximum clear span, the rear room is not. The rear room joists have now been upgraded to 145$\times$47 C24, which have a maximum clear span of 3.1\,m. Since the clear span of the extension is close to the maximum clear span for 120$\times$47 C16, these have now been upgraded to C24. The 120$\times$47 C16 floor joists have also now been upgraded to C24, and the plans have been changed to show them supported by sleeper walls with a maximum separation of 2\,m between supports.
\item {\it The new flat roof should be strapped to masonry walls where joists run parallel, across min 3x timbers with noggins every 1800 centres using 1.2m galvanised steel straps.}\\
If I understand correctly, the only flat roof that this applies to is that over the rear room. The joists of the new flat roof to the study and garage do not run next to and parallel to any masonry walls (see plans). A 1.2\,m galvanised steel strap has been added to the plans of the roof of the rear room.

\end{enumerate}


\begin{thebibliography}{9}
\bibitem{starterkit} \href{https://www.screwfix.com/p/sabrefix-wall-starter-kit-stainless-steel/56037?kpid=56037&cm_mmc=Google-_-Datafeed-_-Building%20and%20Doors?kpid=KINASEKPID&cm_mmc=Google-_-TOKEN1-_-TOKEN2&gclid=EAIaIQobChMIvrnvoZmQgQMVjIFQBh3P7gMrEAQYAiABEgIWrvD_BwE&gclsrc=aw.ds}{Sabrefix Wall Starter Kit Stainless Steel}.



\end{thebibliography}
\end{document}
