\documentclass{extension}

%\usepackage{titling}
%\setlength{\droptitle}{10mm}

\usepackage[superscript]{cite}
\usepackage{hyperref}
\makeatletter
\renewcommand\@citess[1]{\textsuperscript{[#1]}}
\renewcommand\refname{Materials}

\newcommand{\mm}{\,$\mathrm{mm}$}
\newcommand{\m}{\,$\mathrm{m}$}
\newcommand{\uunit}{\,$\mathrm{W\,m^{-2}\,K}$}

\begin{document}

\title{Building Regulations Application \\[2mm]
Response to schedule of additional \\ information/details for: 23/01851/FP}
\author{J. Mullaney}
\date{\today}

\maketitle

\vspace{5mm}

\section{Structure}
\begin{enumerate}
\item {\it Foundations to be to the approval of the District Building Surveyor}\\
I confirm that foundatations for the Porch and Extension will be inspected and approved by the District Building Surveyor prior to the commencement of any further construction. No new foundations are needed for the Utility Room and WC.
\item {\it New walls should be suitably bonded to existing walls. i.e., using a proprietary wall starter kit.}\\
I confirm that new walls will be bonded to the existing walls using a wall starter kit such as SabreFix Stainless Steel Wall Starter Kit, or similar.\cite{starterkit} A note to this effect has been added to page 7 of the plans and has, where necessary, been added to the Supporting Documentation.
\item {\it Wall plates should be strapped to new masonry walls every 2m, using galvanised steel straps at least 1m length.}\\
The only reference to wall plates was on the plans for the porch. It has since been realised that they are not required, and so have been removed from the plans.
\item {\it Can you confirm the roof span? The proposed 120$\times$47 C16 joists appear slightly under spanned according to the Trada tables, at 2.4\,m maximum clear span.}\\
Thank you for pointing out this oversight. The clear span of the extension (study) is 2.36\,m, while the clear span of the rear room is 2.83\,m. So while the extension is just within the maximum clear span, the rear room is not. The rear room joists have now been upgraded to 145$\times$47 C24, which have a maximum clear span of 3.1\,m. Since the clear span of the extension is close to the maximum clear span for 120$\times$47 C16, these have now been upgraded to C24. The 120$\times$47 C16 floor joists have also now been upgraded to C24, and the plans have been changed to show them supported by sleeper walls with a maximum separation of 2\,m between supports.
\item {\it The new flat roof should be strapped to masonry walls where joists run parallel, across min 3x timbers with noggins every 1800 centres using 1.2m galvanised steel straps.}\\
A 1.2\,m galvanised steel strap has been added to the plans of the roof of the rear room and the study. In the case of the study it is tied to an existing masonry wall, while in the case of the rear room it is tied to the new rear wall. In both cases the straps extend over three joists. A plan of the substructure of study roof (showing joists, noggins and strap) has been added to the plans.
\end{enumerate}

\section{Fire safety}
\begin{enumerate}
\item {\it The garage floor level should be a minimum 100\,mm below the floor level to the new utility room.}\\
The utility floor is approx. 293\,mm above the garage floor. This is now shown on the plans and described in Section 3.3.1 of the Supporting Documentation.
\item {\it The fire door to the garage from the utility room requires smoke seals and a self-closing device.}\\
These have now been added to Section 3.3 of the Supporting Documentation and a note to this effect has been added to the plans.
\item {\it The flat roof covering should achieve BROOF(t4) rating.}\\According to the BBA certification (Appendix A) 18\,mm plywood with a fully-adhered 1.2 mm ClassicBond Non-reinforced EPDM membrane achieved ${\rm B_{ROOF}(t4)}$ standard when tested to tested to DD CEN/TS 1187:2012, Test 4 and classified to BS EN 13501-5:2005. To ensure this is achieved, the 18\,mm OSB under the EPDM has been replaced with 18\,mm plywood.
\end{enumerate}

\section{Resistance to moisture and contaminants}
\begin{enumerate}
\item {\it DPC should be 150mm above ground level linked with DPM/radon barrier, and tray DPC provided over the cavity}\\
All DPCs are a minimum of 150\,mm above the ground level. This dimension is now shown on the plans. In the case of the study and porch, the DPC of the inner leaf is linked with the DPM in the solid floor, as shown on the plans. Tray DPCs with stop ends have been added over the window and door openings in the rear wall to the rear room, with two weep holes, separated by no more than 450\,mm, added per opening.
\item {\it Can you clarify the cavity wall insulation? Ie. K106 partial fill cavity board requires minimum 50\,mm clear cavity.}\\
All insulation board in the masonry cavity walls -- front wall of study, \& back wall of rear-room -- has been changed to partial-fill (i.e., Kingspan K108 or similar; K106 is full-fill), with a 50\,mm clear cavity. The U-value calculations have been altered accordingly. All insulation boards in stud walls -- porch, side wall of rear room, wall between rear room and garage -- have been changed to framing boards (i.e., Kingspan K112 or similar). A 25\,mm clear cavity between the brick and stud walls has been added to the side wall. To maintain the wall thickness, the thickness of the insulation has been reduced to 100\,mm, increasing the overall U-value of the wall to $0.33\,{\rm m^2\,K\,W}$. Please note, however, that the heated towel rail has now been removed from the w/c, meaning no part of the rear room will be heated so thermal loss through the side wall will be minimal.
\item {\it A vapour barrier should be fitted to the warm side of the flat roof insulation. A breathable membrane is not required (Cutaway A and B)}\\
This has been amended on the plans.
\item {\it Can you confirm any existing ventilation to the front of the property, and how this will be maintained?}\\
The only existing ventilation to the front of the property (aside from current windows and doors) is a single breather brick underneath the downstairs bay window. This will not be affected by any of the proposed alterations.
\item {\it Note: Background ventilation is required to all new windows/doors, at a rate of 8000mm2 for habitable rooms and kitchens, and 4000mm2 for bathrooms.}\\
Trickle vents will be included in all new windows and doors.
\item {\it Can you confirm how purge ventilation is to be achieved to the study? Ie. opening window or extract mechanical ventilation.}\\
The window will include an opening mechanism to allow purge ventilation. 
\end{enumerate}

\begin{thebibliography}{9}
\bibitem{starterkit} \href{https://www.screwfix.com/p/sabrefix-wall-starter-kit-stainless-steel/56037?kpid=56037&cm_mmc=Google-_-Datafeed-_-Building%20and%20Doors?kpid=KINASEKPID&cm_mmc=Google-_-TOKEN1-_-TOKEN2&gclid=EAIaIQobChMIvrnvoZmQgQMVjIFQBh3P7gMrEAQYAiABEgIWrvD_BwE&gclsrc=aw.ds}{Sabrefix Wall Starter Kit Stainless Steel}.



\end{thebibliography}
\end{document}
